%%%%%%%%%%%%%%%%%%%%%%%%%%%%%%%%%%%%%%%%%%%%%%%%%%%%%%%%%%%%%%%%%%%%%%%%
\chapter{Memoria ekonomikoa}
%%%%%%%%%%%%%%%%%%%%%%%%%%%%%%%%%%%%%%%%%%%%%%%%%%%%%%%%%%%%%%%%%%%%%%%%


     Kapitulu honetan garatutako proiektuak izan duen eragin ekonomikoa aztertzen da. Honetarako erabilitako lanorduen analisia burutu eta proiektuan egindako inbertsioaren itzulkina kalkulatu da. 


%%%%%%%%%%%%%%%%%%%%%%%%%%%%%%%%%%%%%%%%%%%%%%%%%%%%%%%%%%%%%%%%%%%%%%%%
\section{Lanorduen banakatzea}\label{sec:lanen-banakatzea}
%%%%%%%%%%%%%%%%%%%%%%%%%%%%%%%%%%%%%%%%%%%%%%%%%%%%%%%%%%%%%%%%%%%%%%%%
CYCn lanorduak eta bezeroei fakturatzen zaien orduak kudeatu eta jarraitzeko, lanordu bakoitzaren erregistroa egiten da aplikazio propio batean. Ordu bakoitza sailkatu egiten da atazaren arabera.
Sailkapena egiteko hainbat prozesu daude definituta, prozesu batzuk CYCrenak dira eta beste asko bezeroei daude lotuta. Bezeroei inputatutako orduak, hilabete bukaeran fakturatu egiten dira.
Orduen analisia egin den momentuan, 2021eko azaroaren 2tik 2022ko ekainaren 24rako erregistroak jaso eta interpretatu egin dira hitz hauen azpian aurkitzen den \ref{lanen-banakatzea} taulan eta \ref{lanen-piechart} irudiko grafikoan.

Amaiera-datara arteko lanorduetan, 2022ko ekainaren 24tik uztailaren 15era, orain arteko tendentzia berdina jarraitzea espero da: bezeroentzako lana, balorazioak eta banakako formakuntzak. 

\vspace{1cm}
\begin{table}[H]
\centering
\rowcolors{2}{teal!0}{teal!10}
\def\arraystretch{1.5}%
\begin{tabular}{ p{8.5cm} p{2.75cm} p{2cm} }
\hline
\textbf{Ataza} & \textbf{Ordu kopurua} & \textbf{Ehunekoa}\\
\hline
\textit{Bootcamp} formakuntza (\ref{sec:bootcamp} atala) & 428 & \%34 \\
Lanen balorazioa, banakako formakuntza  & 389,75 & \%31 \\
Bezeroentzako lan fakturagarriak  & 281 & \%23 \\
SonarQube, SPFx proiektuetarako (\ref{sonarqube} atala)  & 71,75 & \%6 \\
Absentziak (oporrak, medikua...) & 46,85 & \%4 \\
Bezeroentzako lan ez-fakturagarriak & 12,75 & \%1 \\
Barne lanak (kudeaketa, bilerak...) & 12,25 & \%1 \\
\hline
\textbf{GUZTIRA} & 1242,35 & \%100 \\
\hline
\end{tabular}
\caption{Proiektuaren lanorduen banakatzea, prozesuka}
\label{lanen-banakatzea}
\end{table}
\vfill


\begin{figure}[H]
\vspace{1cm}
\centering
\begin{tikzpicture}
\pie[text = legend]{34/\textit{Bootcamp},
    31/Balorazioak eta formakuntzak,
    23/Lan fakturagarriak,
    6/SonarQube,
    4/Absentziak,
    1/Lan ez-fakturagarriak,
    1/Barne lanak}
\end{tikzpicture}
\caption{Proiektuen lanorduan, sektore-diagraman adieraziak}
\label{lanen-piechart}
\end{figure}


%%%%%%%%%%%%%%%%%%%%%%%%%%%%%%%%%%%%%%%%%%%%%%%%%%%%%%%%%%%%%%%%%%%%%%%%
\section{Proiektuaren bideragarritasuna, enpresaren ikuspegitik}\label{sec:roi}
%%%%%%%%%%%%%%%%%%%%%%%%%%%%%%%%%%%%%%%%%%%%%%%%%%%%%%%%%%%%%%%%%%%%%%%%
Proiektuaren bideragarritasuna ezagutzeko ROI (Return On Investment) neurria kalkulatu egin da. Inbertsio baten eraginkortasuna edo errentagarritasuna ebaluatzeko edo inbertsio ezberdinen eraginkortasuna alderatzeko erabiltzen den errendimendu-neurri bat da \cite{roi}. Inbertsio jakin baten itzulketaren zenbatekoa neurtzen saiatzen da, inbertsioaren kostuaren araberakoa. Neurria kalkulatzeko \ref{roi} irudiko formula erabiltzen da eta ratio edo ehuneko moduan adierazi ohi da. Emaitza positiboa lortzen denean inbertsioa errentagarria dela esan nahi du, eta emaitza geroz eta altuagoa izan errentagarritasun-maila hobea lortu dela adierazten du.

\begin{figure}[H]
\centering
\begin{displaymath}
ROI = \frac{irabaziak - inbertsioa}{inbertsioa}
\end{displaymath}
\caption{ROI neurria kalkulatzeko formula}
\label{roi}
\end{figure}

Inbertsioa edo gastuak kalkulatzerako orduan proiektuak irauten dituen 9 hilabeteak hartu dira kontuan: 2021eko azaroaren 2tik 2022ko uztailaren 16rarte. Diru-sarrerekin ordea, lanorduen banakatzearekin bezala, 2022ko ekainaren 24ra arteko datuak erabili dira. Lanordu bakoitzaren kostua kalkulatzeko ordu bakoitzari 60€ko balioa jarri zaio, industrian ohikoa den prezioa. 

Praktikaren ondorioz eratorritako kideen produktibitate galera ere kontuan hartu da. Esaterako CYCko Carlos Ulibarrena tutorearen inplikazioa eta \textit{Bootcamp} formakuntzan parte hartu duten formatzaileen denbora baita ere. 

\begin{table}[H]
\centering
%%\rowcolors{2}{teal!0}{teal!10}
\def\arraystretch{1.5}
\begin{tabular}{lll}
\hline
\multirow{6}{*}{\textbf{Inbertsioa}} & \cellcolor{teal!10}Ordenagailua eta osagarriak & \cellcolor{teal!10}934,9 €   \\
                            & Microsoft 365 lizentzia     & 167,4 €   \\
                            & \cellcolor{teal!10}\textit{Bootcamp}aren prestakuntza & \cellcolor{teal!10}500 €     \\
                            & Tutorearen jarraipena       & 300 €     \\
                            & \cellcolor{teal!10}Soldata & \cellcolor{teal!10}6800 €    \\ \cline{2-3} 
                            & \underline{\textit{GUZTIRA}}           & 8702,3 €  \\
\hline
\multirow{2}{*}{\textbf{Diru-sarrerak}}  & \cellcolor{teal!10}Bezeroei fakturatutako lana & \cellcolor{teal!10}16860 €   \\ \cline{2-3}     
                            & \underline{\textit{GUZTIRA}}               & 16860 €   \\
\hline
\multicolumn{2}{l}{\textbf{Irabaziak}}& 8157,7 € \\  
\hline
\multicolumn{2}{l}{\cellcolor{teal!25}\textbf{ROI}}& \cellcolor{teal!35}0,93 \\  
\hline
\end{tabular}
\caption{Proiektuaren ROIaren estimazioa eta erabilitako datuak}
\label{roi-table}
\end{table}

\ref{roi-table} taulako \%93ko ROI emaitzari erreparatuz, errentagarritasun argia dagoela esan daiteke.
Irabazi guztiak bezeroei fakturatutako lanetik lortu dira eta \ref{sec:lanen-banakatzea} ataleko datuetan ikus daiteken moduan ia denboraren laurdena erabili da lan hauetan.
Aipatu behar da kudeaketarako gastuak, bulegoaren kostua eta bestelako gastuak ez direla jaso egindako analisian eta honek inpaktu negatiboa duela ROIean. 

