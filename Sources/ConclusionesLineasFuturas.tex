%%%%%%%%%%%%%%%%%%%%%%%%%%%%%%%%%%%%%%%%%%%%%%%%%%%%%%%%%%%%%%%%%%%%%%%%
\chapter{Conclusiones y líneas futuras}
%%%%%%%%%%%%%%%%%%%%%%%%%%%%%%%%%%%%%%%%%%%%%%%%%%%%%%%%%%%%%%%%%%%%%%%%

\iffalse
     Kapitulu honetan proiektuaren eta praktiken ondorioak jasotzen dira. Atal bakoitzean
     ikuspuntu ezberdinetatik azaltzen eta sakontzen da garatutako atazen inguruko gogoeta.



%%%%%%%%%%%%%%%%%%%%%%%%%%%%%%%%%%%%%%%%%%%%%%%%%%%%%%%%%%%%%%%%%%%%%%%%
\section{Ondorio ekonomikoak}
%%%%%%%%%%%%%%%%%%%%%%%%%%%%%%%%%%%%%%%%%%%%%%%%%%%%%%%%%%%%%%%%%%%%%%%%
\ref{sec:roi} atalean azaldu bezala, proiektuaren garapena ekonomikoki emankorra izan da. Nahiz eta formakuntzan eta lanerako materialean inbertsioa handia izan, bezero ezberdinetarako lanak burutu dira eta zerbitzu eta proiektu berriak eskaintzeko aukera berriak aztertu eta eraiki dira. Proiektuan zehar egindako lanen zati bat etorkizunean berrerabiltzeko eta probetxua ateratzeko aukera garbia dago.
%%%%%%%%%%%%%%%%%%%%%%%%%%%%%%%%%%%%%%%%%%%%%%%%%%%%%%%%%%%%%%%%%%%%%%%%
\section{Garapen Jasangarrirako Helburuekiko ondorioak}
%%%%%%%%%%%%%%%%%%%%%%%%%%%%%%%%%%%%%%%%%%%%%%%%%%%%%%%%%%%%%%%%%%%%%%%%
\acrfull{gjh} pertsonen, planetaren eta oparotasunaren aldeko ekintza-plan bat da, bake unibertsala eta justiziarako sarbidea indartzeko asmoa daukana \cite{gjh}. Nazio Batuen Erakundeak (\acrshort{nbe}) sortutako 17 helburu, 169 xede eta 231 adierazlez dago osatua eta Agenda 2030 planaren barruan aurkitzen da.

Proiektuan zehar garatutako emaitzek \acrshort{gjh}kiko duten inpaktua neurtzeko SDG Impact Assessment Tool\footnote{SDG Impact Assessment Tool \url{https://sdgimpactassessmenttool.org}}  tresna erabili da. Tresna honen bidez, helburu bakoitzerako inpaktua eta motibazioa deskribatu da. Informazio guzti honekin, \ref{gjh-guztiak} irudiko taula edo sailkapena lortu da. Helburu gehienek ez dute inpakturik jasaten garatutako proiektutik, baina badaude helburu batzuk zuzenean inpaktu positiboa dutenak, \ref{gjh-positiboak} irudian agertzen direnak alegia.

\begin{figure}[H]
\centering
\includegraphics[width=0.95\textwidth]{Figures/ondorioak/gjh-guztiak.png}
\caption{SDG Impact Assessment Tool tresnan lortutako emaitza}
\label{gjh-guztiak}
\end{figure}

\begin{figure}[H]
\centering
\includegraphics[width=0.8\textwidth]{Figures/ondorioak/gjh-positiboak.png}
\caption{Eragin zuzen positiboa lortu den helburuak}
\label{gjh-positiboak}
\end{figure}

Proiektuaren emaitzei erreparatuz (\ref{sec:emaitzak} atala), iradokizunen kudeaketari dagokion tresna da \acrshort{gjh}etan inpaktu gehien duena. Tresna honi esker lantokian egin daitezken hobekuntzak jaso, kudeatu eta exekutatzen dira. Hobekuntza hauek alor ezberdinetara egon daitezke zentratuta: segurtasun arriskuak minimizatzea, ekidin daitezken materialak kentzea, hondakin gutxiago sortzea, prozesu jasangarriagoak lortzeko urratsak hartzea... Tresna guztiz digitala denez, informazio sarrera eta jakinarazpen guztiak digitalki egiten dira eta paperaren erabilera minimora jaitsi da. Soluzioak 3., 12. eta 15. helburuetan du inpaktu zuzena: osasuna eta ongizatea, ekoizpen eta kontsumo arduratsuak, lehorreko ekosistemetako bizitza. 

Proiektu osoak langileentzako plataforma digital bat izan du ardatz eta industria sektoreko erakunde baten barruan egin da. Erakundearentzat aldaketa handia izan, kolaboraziorako aukerak maximizatuz eta komunikazioa erraztuz. Lanerako modua eraldatzeagatik 9. helburuan jaso da inpaktu positibo zuzena, industria, berrikuntza eta azpiegiturarena. 



SDG Impact Assessment Tool tresnaren bidez lortutako emaitza osoa eta egindako iruzkinak \ref{app:gjh} eranskinean aurkitzen dira. 

\fi