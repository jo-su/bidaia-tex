%%%%%%%%%%%%%%%%%%%%%%%%%%%%%%%%%%%%%%%%%%%%%%%%%%%%%%%%%%%%%%%%%%%%%%%%
\chapter{Balorazio pertsonala}
%%%%%%%%%%%%%%%%%%%%%%%%%%%%%%%%%%%%%%%%%%%%%%%%%%%%%%%%%%%%%%%%%%%%%%%%


     Kapitulu honetan proiektuaren eta praktiken balorazio pertsonala azaltzen da. Bai pertsonalki, bai profesionalki lana egiteak izan duen eragina.

\newacronym{lip}{LIP}{Lan-Ikaste Partekatzea}
\newacronym{gbl}{GBL}{Gradu Bukaerako Lana}
\newacronym{scss}{SCSS}{Sassy Style Sheets}
\newacronym{html}{HTML}{HyperText Markup Language}
\newacronym{css}{CSS}{Cascading Style Sheets}

Praktiken garapena esperientzia oso positiboa izan da alderdi guztietatik begiratuta. Ez da izan nire lehen lan esperientzia, azkeneko ikasturteetan zehar lan-ikaste partekatze (\acrshort{lip}) modalitatean egon naizelako, lanaldi-erdian praktikak egiten. Hala ere, proiektuarekin eta gradu bukaerako lan (\acrshort{gbl}) honekin hastean jauzi handia eman dudala eta ahaldundu naizela nabaritu dut. 

Hasierako formakuntza oso aberasgarria iruditu zait, gogorra eta trinkoa baita ere. Zati teoriko berriak ikasi edo unibertsitateko jakintzak sendotu ditut. Jakintza hauek ariketa praktikoetan islatu ditut baita ere. Departamentuko kide bihurtu nahiz ikasitako oinarriekin eta bezero errealetan kontzeptu ezberdinak aplikatzeko aukerak eduki ditut. 

Denetariko lanak egin ditut eta prozesuan zehar eginkizun anitzetako trebetasunak eskuratu ditut: teknikoki software berriak garatzetik, bezeroen betekizunak jaso, proiektuak eraiki eta zalantzak erantzutera. Pertsonekiko komunikazioa eta lanak koordinatzeko gaitasuna lortu dut, eta esperientziarekin hobetuz noa. Izan ere, ikasteko modu bakarra egoera errealetan lan egitea da. Inguruko lankideek urteko esperientziaz lortutako intuizioa neureganatzen saiatzen naiz posible dudanean.   

Teknikoki asko ikasi dut, eta unibertsitate eskoletan landutako teknologiak ezagutu eta menperatzeko gai izan naiz. Unibertsitateko azken urteetan ikasitako HTML, CSS eta JavaScriptetik abiatuta,  TypeScript, React liburutegiarekin eta SCSSrekin lan egin dut oraingoan. Web teknologia hauekin lan egiteak izugarri gustatu zait, eta etorkizunera begira oso interesgarriak diren trebetasunak ezagutu ditut.

Pertsonalki ere, eguneroko bulegoko giroa atsegina izan da eta lankideen arteko harremana paregabekoa da. Zalantzak sortzean gertu sentitu ditut laguntzeko prest dauden pertsonak, eta nik aukera izan dudanean berdin egin dut nik. 

Gaineko lerroetan aipatutakoagatik, proiektuan zehar motibatuta sentitu naiz eta lan bikaina egiteko indarra eta gogoa izan dut. Bizi eta egondako egoera eta inguruneagatik ez balitz, posible izango ez litekeena. 