\newglossaryentry{API}{%
  name        = {\textbf{API}},
  description = {Aplikazioen softwarea integratzen eta garatzeko erabiltzen diren definizio eta protokoloen. multzoa, bi softwareren arteko komunikazioa ahalbidetuz.\\}
}

\newglossaryentry{localstorage}{%
  name        = {\textbf{Localstorage}},
  description = {Nabigatzaileen propietate bat da non JavaScript gune eta aplikazioek iraungitzen ez diren gako-balio pareak gordetzen dituzte.\\}
}

\newglossaryentry{frontend}{%
  name        = {\textbf{Frontend}},
  description = {Aplikazio batean, erabiltzaileek ikusiko duten elementu eta ezaugarrien kodetzean eta sorkuntzan zentratzen den programazio mota.\\}
}

\newglossaryentry{backend}{%
  name        = {\textbf{Backend}},
  description = {Aplikazio batean, erabiltzaileek ikusten ez duten aldean zentratzen den programazio mota. Hau da, zerbitzarian exekutatzen den kodean garapena da eta aplikazioak interakzioa ahalbidetzen duena.\\}
}

\newglossaryentry{open-source}{%
  name        = {\textbf{Kode irekia}},
  description = {Modu askean, publikoan, banatu eta garatzen den softwarea. Edozeinek ikusi, aldatu eta banatu dezakeen kodea da.\\}
}

\newglossaryentry{testing}{%
  name        = {\textbf{\emph{Testing}a}},
  description = {Software produktu batek betekizunak betetzen dituela, esperotako jarrera jarraitzen duela eta defekturik ez duela ziurtatzeko metodoa da.\\}
}

\newglossaryentry{caching}{%
  name        = {\textbf{\emph{Cache}atzea}},
  description = {Fitxategien edo datuen kopiak \emph{cache} batean edo aldi baterako biltegiratzeko lekuan gordetzeko prozesua da, azkarrago aurkez daitezen.\\}
}