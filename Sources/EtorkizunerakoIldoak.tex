%%%%%%%%%%%%%%%%%%%%%%%%%%%%%%%%%%%%%%%%%%%%%%%%%%%%%%%%%%%%%%%%%%%%%%%%
\chapter{Etorkizunerako ildoak}
%%%%%%%%%%%%%%%%%%%%%%%%%%%%%%%%%%%%%%%%%%%%%%%%%%%%%%%%%%%%%%%%%%%%%%%%


     Kapitulu honetan garatutako proiektuak etorkizunean izan dezaken bilakaeraz jarduten da eta proiektuaren iraupenaren barruan burutu ezin izan diren lanen azalpena ematen da.


Proiektuan eraikitako soluzioak eboluzionatzeko aukera argiak ikusten dira. Ataza batzuk posible izan ez badira, bezeroen aldetik beharrezko ekintzak burutu ez direlako izan da. Besteak, denbora faltagatik edo lehentasunak beste lanetan ezartzeagatik izan dira. 

Iradokizunen soluzioarentzat jada definituta ditugu etorkizunerako lanak:
\begin{itemize}
    \item Formularioetan erabiltzaileen informazio osoa sartu ordez, Microsoften Direktorio Aktibotik datu hauek lortzea eta erabiltzea. Hau lortzeko, lehendabizi erabiltzaileen datuak zuzen eta osorik jasota egon behar dira.  
    \item Jakinarazpen eta ekintza automatiko gehiago egiten dituzten Power Automate fluxuak eraikitzea. 
    \item Iradokizunen inguruko estatistikak kalkulatzea eta \textit{dashboard} edo panel batean datuak erakustea grafiko eta indikatzaileen bidez.
    \item Erabiltzaileek iradokizunak egitean lortzen dituzten puntuen jarraipena eta kudeatzeko funtzionalitatea. 
\end{itemize}

Zerrenda bateko elementuak esportatzeko luzapenari probetxu handia ateratzeko aukera dago, bezero ezberdinetan garatu dezakeguna. Proiektuan azaldutako komandoetatik haratago dauden funtzionalitate berriak gehitzea dago aurreikusia: 
\begin{itemize}
    \item Zerrenda zehatzetan bakarrik ikusgarri dauden komandoak inplementatzea.
    \item Aurretik definitutako txantiloi baten arabera PDF dokumentuak sortzea.  
    \item Elementu baten eranskinetan aurkitzen diren irudiak jasotzen dituen PDF dokumentuak sortzea. 
\end{itemize}

\textit{Testing}aren inguruan ere askoz gehiago lan egin daiteke. \textit{End-to-end} motako probak egiteko Cypress erabili dugu, baina baliokidea den Playwright\footnote{Playwright \url{https://playwright.dev}} tresna erabiltzea ere interesgarria izan daiteke. Test unitarioak egitea ere gomendagarria izan daiteke,  batez ere bezero ezberdinen artean berrerabiltzen diren soluzioak garatzerako orduan.