%%%%%%%%%%%%%%%%%%%%%%%%%%%%%%%%%%%%%%%%%%%%%%%%%%%%%%%%%%%%%%%%%%%%%%%%
\chapter{Ondorioak}
%%%%%%%%%%%%%%%%%%%%%%%%%%%%%%%%%%%%%%%%%%%%%%%%%%%%%%%%%%%%%%%%%%%%%%%%


     Kapitulu honetan proiektuaren eta praktiken ondorioak jasotzen dira. Atal bakoitzean
     ikuspuntu ezberdinetatik azaltzen eta sakontzen da garatutako atazen inguruko gogoeta.



%%%%%%%%%%%%%%%%%%%%%%%%%%%%%%%%%%%%%%%%%%%%%%%%%%%%%%%%%%%%%%%%%%%%%%%%
\section{Ondorio teknikoak}
%%%%%%%%%%%%%%%%%%%%%%%%%%%%%%%%%%%%%%%%%%%%%%%%%%%%%%%%%%%%%%%%%%%%%%%%
Teknikoki, proiektuan garatutako lanek software eta zerbitzuen garapenerako tresna aitzindariak erabili dira.
Proiektuaren garapenean zehar Power Automate eta Power Apps \textit{no-code} edo \textit{low-code} tresnen erraztasunaz eta bizkortasunak baliotu da soluzioak denbora errekorrean inplementatu eta probatzeko.
Bezeroen betekizunak teknikoki aurreratuak eta pertsonalizazio handia behar izan dutenean Microsoft 365 ingurunearekin integratzen diren funtzionalitateak eraiki egin dira SharePoint Frameworkaz baliatuz.

Azpimarratzekoa da kode irekiaren inguruko komunitateak duen garrantzia garatu diren soluzioen inguruan. Pentsaezina izango litzake lortutako emaitzak garatzea komunitateak sortutako osagarriak izango ez bagenitu.
SharePoint Framework adibidez, node.js eta React kode irekiko liburutegian daude oinarrituta. Microsoft 365 PnP komunitateak ere dozena bat tresna mantentzen ditu garatzaileen eta erabiltzaileen onurarako eta proiektuetan birritan erabiltzen direnak. 

Sarreran definitutako helburuak kontuan izanda, behar ezberdinei erantzuna emateko soluzioak eraiki dira, bezeroen egoerarako espresuki inplementatu direnak eta erakunde eta partaide ezberdinen ikuspuntua eta nahiak kontuan hartuz.

%%%%%%%%%%%%%%%%%%%%%%%%%%%%%%%%%%%%%%%%%%%%%%%%%%%%%%%%%%%%%%%%%%%%%%%%
\section{Ondorio metodologikoak}
%%%%%%%%%%%%%%%%%%%%%%%%%%%%%%%%%%%%%%%%%%%%%%%%%%%%%%%%%%%%%%%%%%%%%%%%
Lanak sortzerako momentuan jarraitutako metodologian, bezeroetako partaide anitzekin koordinatu da eta kontaktu zuzena egon da. Departamentuko kideetatik eta CYCko beste departamentuetatik \textit{feedback}a jaso eta lortutako ezagutza erabili da soluzioa beharretara eta testuingurura bideratzeko. 

Lantalde osoaren eskuragarritasuna izatea ezinbestekoa izan da erabaki zuzenak hartzeko. Jakintza kolektibo honi esker, urtetako esperientzia eta Microsoft 365en azken berrikuntza eta teknologien inguruko  formakuntzan ikasitakoa konbinatuz, arrakasta lortu da lortutako emaitzetan.

Lan taldearen kolaboratzeko moduari esker proiektu anitz eta ezberdinetan inplikatzeko aukera egon da. Sarritan formakuntza eta ezagutzak eskualdatzeko saioak egin dira eta etengabeko ikasketa egiteko beharra
argi ikusi da. Batez ere, egunero aldatzen diren plataforma eta zerbitzuekin lan egiten denean. 
%%%%%%%%%%%%%%%%%%%%%%%%%%%%%%%%%%%%%%%%%%%%%%%%%%%%%%%%%%%%%%%%%%%%%%%%
\section{Osasun eta segurtasunarekiko ondorioak}
%%%%%%%%%%%%%%%%%%%%%%%%%%%%%%%%%%%%%%%%%%%%%%%%%%%%%%%%%%%%%%%%%%%%%%%%
Lantoki seguru eta osasuntsuan lan egiteko orduan, lan arriskuen inguruko formakuntza jasotzea aparte neurri eta gomendio ezberdinak jarraitzeko materiala eskuratu da, baldintzen agiria azaltzen den \ref{sec:baldintzak} atalean azaltzen direnak. Lan ordutegian ere beharrezko atsedenaldiak eta lanorduak ezarri dira, buruak arnas har dezan eta gorputzak mugitzeko aukera izateko. 

Ohitura osasungarriak jarraitzeko urtaroro denboraldiko fruta eduki da eskura bulegoan eta hainbat saio antolatu dira osasun mentala eta honen inguruko arazoak kudeatzeko informazioarekin, lan egoeretan fokua jarriz.
%%%%%%%%%%%%%%%%%%%%%%%%%%%%%%%%%%%%%%%%%%%%%%%%%%%%%%%%%%%%%%%%%%%%%%%%
\section{Ondorio ekonomikoak}
%%%%%%%%%%%%%%%%%%%%%%%%%%%%%%%%%%%%%%%%%%%%%%%%%%%%%%%%%%%%%%%%%%%%%%%%
\ref{sec:roi} atalean azaldu bezala, proiektuaren garapena ekonomikoki emankorra izan da. Nahiz eta formakuntzan eta lanerako materialean inbertsioa handia izan, bezero ezberdinetarako lanak burutu dira eta zerbitzu eta proiektu berriak eskaintzeko aukera berriak aztertu eta eraiki dira. Proiektuan zehar egindako lanen zati bat etorkizunean berrerabiltzeko eta probetxua ateratzeko aukera garbia dago.
%%%%%%%%%%%%%%%%%%%%%%%%%%%%%%%%%%%%%%%%%%%%%%%%%%%%%%%%%%%%%%%%%%%%%%%%
\section{Garapen Jasangarrirako Helburuekiko ondorioak}
%%%%%%%%%%%%%%%%%%%%%%%%%%%%%%%%%%%%%%%%%%%%%%%%%%%%%%%%%%%%%%%%%%%%%%%%
\newacronym{gjh}{GJH}{Garapen Jasangarrirako Helburuak}
\newacronym{nbe}{NBE}{Nazio Batuen Erakundea}
\acrfull{gjh} pertsonen, planetaren eta oparotasunaren aldeko ekintza-plan bat da, bake unibertsala eta justiziarako sarbidea indartzeko asmoa daukana \cite{gjh}. Nazio Batuen Erakundeak (\acrshort{nbe}) sortutako 17 helburu, 169 xede eta 231 adierazlez dago osatua eta Agenda 2030 planaren barruan aurkitzen da.

Proiektuan zehar garatutako emaitzek \acrshort{gjh}kiko duten inpaktua neurtzeko SDG Impact Assessment Tool\footnote{SDG Impact Assessment Tool \url{https://sdgimpactassessmenttool.org}}  tresna erabili da. Tresna honen bidez, helburu bakoitzerako inpaktua eta motibazioa deskribatu da. Informazio guzti honekin, \ref{gjh-guztiak} irudiko taula edo sailkapena lortu da. Helburu gehienek ez dute inpakturik jasaten garatutako proiektutik, baina badaude helburu batzuk zuzenean inpaktu positiboa dutenak, \ref{gjh-positiboak} irudian agertzen direnak alegia.

\begin{figure}[H]
\centering
\includegraphics[width=0.95\textwidth]{Figures/ondorioak/gjh-guztiak.png}
\caption{SDG Impact Assessment Tool tresnan lortutako emaitza}
\label{gjh-guztiak}
\end{figure}

\begin{figure}[H]
\centering
\includegraphics[width=0.8\textwidth]{Figures/ondorioak/gjh-positiboak.png}
\caption{Eragin zuzen positiboa lortu den helburuak}
\label{gjh-positiboak}
\end{figure}

Proiektuaren emaitzei erreparatuz (\ref{sec:emaitzak} atala), iradokizunen kudeaketari dagokion tresna da \acrshort{gjh}etan inpaktu gehien duena. Tresna honi esker lantokian egin daitezken hobekuntzak jaso, kudeatu eta exekutatzen dira. Hobekuntza hauek alor ezberdinetara egon daitezke zentratuta: segurtasun arriskuak minimizatzea, ekidin daitezken materialak kentzea, hondakin gutxiago sortzea, prozesu jasangarriagoak lortzeko urratsak hartzea... Tresna guztiz digitala denez, informazio sarrera eta jakinarazpen guztiak digitalki egiten dira eta paperaren erabilera minimora jaitsi da. Soluzioak 3., 12. eta 15. helburuetan du inpaktu zuzena: osasuna eta ongizatea, ekoizpen eta kontsumo arduratsuak, lehorreko ekosistemetako bizitza. 

Proiektu osoak langileentzako plataforma digital bat izan du ardatz eta industria sektoreko erakunde baten barruan egin da. Erakundearentzat aldaketa handia izan, kolaboraziorako aukerak maximizatuz eta komunikazioa erraztuz. Lanerako modua eraldatzeagatik 9. helburuan jaso da inpaktu positibo zuzena, industria, berrikuntza eta azpiegiturarena. 



SDG Impact Assessment Tool tresnaren bidez lortutako emaitza osoa eta egindako iruzkinak \ref{app:gjh} eranskinean aurkitzen dira. 