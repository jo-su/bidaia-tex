\begin{center}
  {\Large \textsc{\textcolor{teal}{Laburpena}}}
\end{center}
%
\noindent
%
Eraldaketa digitalak sektore eta alor guztietan izan du inpaktua, eta emaitzak izugarriak izan dira. Hala ere, enpresa askoren lan egiteko modua guztiz zaharkitua da eta lankideen arteko komunikazioa, kolaborazioa eta produktibitatea urria edo hobetu daitekeena da. Nola lor daiteke erakunde baten komunikazioak, datuak eta prozesuak jasotzen dituen plataforma integratua inplementatzea eta langileek erabiltzea? Sektorean liderra den Microsoft 365 zerbitzuak SharePoint eta Microsoft Teams tresnak eskaintzen dituzte langileen arteko elkarguneak sortzeko eta laneko ataza guztiak gune hauetatik bideratzeko. Soluzio hauek erakunde bakoitzaren neurrira eta kontestura moldatzen dira, erabiltzeko errazak eta intuitiboak izan daitezen. Plataformaren portaera natiboari, kodez garatutako luzapenak eta \textit{no-code} edo \textit{low-code} tresnekin eraikitako funtzionalitateak gehituz, enpresari eta langileei orientatutako esperientzia pertsonalizatuak sortzen dira. Negozioaren eta langileen egoerak eboluzionatu ahala, tresna hauek baita ere bilakatu egingo dira momentuko egoerari erantzuna emateko. Eraikitako soluzioei esker lanak optimizatzeaz eta errazteaz gain, negozio helburuak betetzeko gaitasuna ematen da. 

\textsc{\textcolor{teal}{Hitz gakoak:}} eraldaketa digitala, langileen gune adimendunak, komunikazioa, kolaborazioa, Microsoft 365, SharePoint, Microsoft Teams

\vspace{2cm}

\begin{center}
  {\Large \textsc{\textcolor{teal}{Abstract}}}
\end{center}
%
\noindent
%
Digital transformation has affected all sectors and fields, and the results are enormous. However, the way many companies work is completely antiquated and communication, collaboration and productivity among colleagues is poor and can be improved. How can a company implement an integrated platform that includes communications, data, and processes that all employees use? 
The industry-leading Microsoft 365 service provides with tools such as SharePoint and Microsoft Teams that help create meeting points where all business-related content is shared. These solutions have to be catered to each company, so that they are intuitive and easy to use. By adding code-developed extensions and functionalities created with \textit{no-code} or \textit{low-code} tools to the platform's native behaviour, we create experiences customised to the business and its employees. As the situation of the company and workers evolve, these tools will also adapt to current needs. In addition to optimizing and facilitating jobs through the construction of solutions, they provide the ability to meet business goals.


\textsc{\textcolor{teal}{Keywords:}} digital transformation, smart digital workplaces, communication, collaboration, Microsoft 365, SharePoint, Microsoft Teams

\newpage

\begin{center}
  {\Large \textsc{\textcolor{teal}{Resumen}}}
\end{center}
%
\noindent
%
La transformación digital ha afectado a todos los sectores y campos, y los resultados han sido enormes. Sin embargo, la forma en que trabajan muchas empresas es completamente anticuada y la comunicación, colaboración y productividad entre compañeros de trabajo es deficiente y se puede mejorar. ¿Cómo puede una empresa implementar una plataforma integrada que incluya comunicaciones, datos y procesos que utilicen todos los empleados?
El servicio de Microsoft 365, líder en la industria, proporciona herramientas como SharePoint y Microsoft Teams que ayudan a crear puntos de encuentro donde se comparte todo el contenido relacionado con el negocio. Estas soluciones se construyen a medida de cada empresa, para que sean intuitivas y fáciles de usar. Al agregar extensiones desarrolladas con código y funcionalidades creadas con herramientas \textit{no-code} o \textit{low-code} al comportamiento nativo de la plataforma, creamos experiencias personalizadas para la empresa y sus empleados. A medida que evolucione la situación del negocio y de los trabajadores, estas herramientas también se adaptarán a las necesidades actuales. Las soluciones construidas no sólo permiten optimizar y facilitar los trabajos, sino que también capacitan para cumplir con los objetivos de negocio.

\textsc{\textcolor{teal}{Palabras clave:}} transformación digital, áreas de trabajo digitales, comunicación, colaboración, Microsoft 365, SharePoint, Microsoft Teams
