%%%%%%%%%%%%%%%%%%%%%%%%%%%%%%%%%%%%%%%%%%%%%%%%%%%%%%%%%%%%%%%%%%%%%%%%
\chapter{Sarrera}
%%%%%%%%%%%%%%%%%%%%%%%%%%%%%%%%%%%%%%%%%%%%%%%%%%%%%%%%%%%%%%%%%%%%%%%%


     Kapitulu honetan proiektua ulertu ahal izateko hainbat azalpen ematen dira. Alde batetik proiektuak soluzioa eman nahi dien arazoa azaltzen da, baita ere zein den industrian dauden ikuspegi ezberdinak honen inguruan. Bestetik proiektua garatzeko plangintza azaltzen da eta azkenik proiektua burutzeko beharrezkoa diren materiala eta baliabideak aurkezten dira baldintzen agirian.


%%%%%%%%%%%%%%%%%%%%%%%%%%%%%%%%%%%%%%%%%%%%%%%%%%%%%%%%%%%%%%%%%%%%%%%%
\section{Arazoa}
%%%%%%%%%%%%%%%%%%%%%%%%%%%%%%%%%%%%%%%%%%%%%%%%%%%%%%%%%%%%%%%%%%%%%%%%
Lan egiteko modua aldatu da, baita ere langileen espektatibak eta beharrak lan egiterako 
orduan. Langileek edozein lekutatik eta edozein gailutatik lan egin ahal izatea espero dute. 
Covid-19 pandemiak eragindako osasun larrialdiak, langileen kontziliazio beharrak eta telelanak lan ingurunea normaltasun berrira moldatzera behartu dute \cite{mckinsey-survey}.

Enpresentzak informazioaren antolakuntza, komunikazioa eta kolaborazioa beharrezkoa
dira eguneroko lanetan. 200 urtez erakundeek lantoki fisikoak eraiki dituzte,
ezagutu eta ulertzen ditugunak. Baina lanerako kontestuak izugarrizko eboluzioa bizi 
izan du eta orain lantoki digitalak, \emph{digital workplace} ere deituak, diseinatu eta hauei forma eman behar zaie. Plataforma hauek langileak, kolaboratzaileak eta bezeroak batzea lortzen dute.

Lantoki digitalen ikuspegiak lantoki fisikoen elementu guztien baliokideak jasotzen 
ditu: bileretarako, eztabaidetarako, produktibitaterako eta interakzio sozialetarako
tresnak esaterako. Lantoki digitalaren kontzeptuak ezagutzen ditugun intranetetatik
haratago doaz. Ez da proposatzen intranetak ''handitzea'' lantoki digital bihurtzeko, 
eboluzionatzea baizik mezularitza, bideo, kolaborazio tresnak eta aplikazioak gehituz.
Ikuspegi estrategikoa aldatu behar da eta intraneta lantoki digitalaren konponente bat 
bihurtzen da \cite{dwg-from-intranet-to-digital-workplace}.

Modu batean, aldaketa hauek onerako dira. Langileek, lanerako tresna egokiekin, edozeinekin eta edonondik lan egin dezakete. Edonondik lan egin ahal izatea da lantokien transformazioaren etorkizuna. Forrester-en arabera \cite{forrester-anywhere-work-strategy}, enpresen \%60ak lan egiteko modelo hibridoa erabiliko du aurrerantzean. Hau da, bulegoak eta lantegi fisikoak mantenduko dira baina edonondik lan egiteko aukera ere eskainiko da langileei. Honen helburua produktibitatearen onurak maximizatzea eta langileen asebetetze maila handitzea dira, lanerako hitzarmen malguen bitartez. Egoera berri honek erronka berriak dakartza negozioen liderrentzat.

Produktibitatearen inguruan, askok galdetzen diogu geure buruari ea nola den posible denok lanpetuta egotea eta era berean lanak aurrera egiten ez duela sentitzea. Lana aurrera eraman ahal izateko interakzio ezberdinak egiten ditugu, eta hauetako batzuk zentzurik ez dituztenean lana atzeratu egiten da. Lan eredu hibridorako bidean, interakzio asko aldatu, agertu edo desagertu daitezke. Trantsizio arrakastatsua lortzeko, interakzio hauek sailkatu eta aztertu behar dira ondoren soluzio teknologiko ezberdinetik eta automatizazioaren bitartez optimizatu ahal izateko \cite{mckinsey-interactions}.

Aldaketekiko erresistentzia handia egon daiteke hasiera batean bai zuzendarien artean, baita langileen artean ere. Produktibitate maila mantentzea, erakundearen kultura jarraitzea edo segurtasuna eta datuen pribatutasuna dira erronka esanguratsuenak. Erronkei aurre egiteko sormenezko irtenbideak behar dira. Komunikazio eta kolaborazio tresna bateratuak bezalako konponbide teknologikoek, sareko azpiegitura fidagarri eta seguruekin konbinatuta, oztopo horiek gainditzen lagun dezakete eta lantoki digitalak ahalbidetzen dituen aukerak maximizatzen.

Askorentzat ezaguna da Maslow piramidea \cite{wiki-maslow}, Abraham Maslow psikologoak garatutako gizakiaren motibazioei buruzko teoria. Teoria honek pertsonen beharrak premiaren arabera sailkatzen ditu. Piramidearen oinarrian behar fisiologikoak aurkitzen dira, bizitzeko ezinbestekoak direnak eta piramidean gorantz joan ahala beharren premia jaitsiz doa gailurrera iritsi arte, autoerrealizazioa.

Kontzeptu berdina jarrai dezakegu enpresaren beharrei erreparatuz, \ref{img:maslow} irudian azaltzen dena. Oinarrian kostuak murriztea eta irabaziak sortzea ditugu, gorago langile eta bezeroen asebetetzea eta azkenik tontorrean elkarlana, kultura berritzailea eta antolaketaren arintasuna ditugu.

\begin{figure}[h]
\centering
\includegraphics[width=0.7\textwidth]{Figures/sarrera/Maslows-Hierarchy-of-Enterprise-2.0-Needs.jpg}
\caption{Maslowren hierarkia enpresa ikuspegitik}
\label{img:maslow}
\end{figure}

Lantoki digitala ardatz da behar ezberdinak asetzeko. Kostuak murriztuz prozesu errepikakorrak automatizatuz, aukera berriei etekina ateraz, pertsonen nahiak betetzen dituzten soluzioekin edo kolaboratzeko tresnak martxan jarriz, besteak beste.

%%%%%%%%%%%%%%%%%%%%%%%%%%%%%%%%%%%%%%%%%%%%%%%%%%%%%%%%%%%%%%%%%%%%%%%%
\section{Aurrekariak eta artearen egoera}
%%%%%%%%%%%%%%%%%%%%%%%%%%%%%%%%%%%%%%%%%%%%%%%%%%%%%%%%%%%%%%%%%%%%%%%%

Eraldaketa digitala teknologia inplementatzeko eta honen onurak ustiatzeko enpresa batek hartzen dituen urrats eta ekintzak dira. Gaur egungo negozio-eragiketak
ulertuz hasten da prozesua eta teknologiaren bidez nola eboluzionatu daitezken
aztertzen da. 

Ezin da ahaztu transformazio digitalerako estrategiak hainbat saiakera, erabiltzaileen \textit{feedback}-saio eta denbora behar dutela emaitza positiboak lortzeko. Langile batzuentzat ''mundu digitalera'' trantsizionatzea erronka 
handia izan daiteke edo aldaketekiko erresistentzia erakusten dute. Konpainiek oztopo 
hauek gainditzeko plangintza sendo batekin erantzun behar dute. Maila eta mota 
guztietako langileen inplikazioa, liderren motibazioa eta lan-taldearen prestakuntza 
ezinbestekoak dira emaitza arrakastatsuak lortzeko. 

Azkeneko urteetako gertakariek telelana edo lan hibridoaren lan modalitatera moldatzera behartu ditu erakunde askori. Momentu honetan agerian gelditu dira enpresen plataformen gabeziak eta honen ondorioz langileentzako esperientziak sortzeko eskaera inoiz baino handiagoa da. 

Intranet plataformen egoera eta merkatua nola dagoen ezagutzeko Forrester agentziak argitaratutako \textit{The Forrester Wave™: Intranet Platforms, Q1 2022} txostena \cite{forrester-wave} hartu da erreferentzia modura. Txosten honetan intranet plataformak eskaintzen dituzten 12 enpresa eta produktu esanguratsuenak aztertzen dira: Akumina, Beezy, COYO, Igloo Software, Interact, LiveTiles, LumApps, MangoApps, Microsoft, Powell Software, Simpplr eta Unily. Enpresa hauen eskaintza, estrategia eta merkatu-tamaina konparatu dira talde ezberdinetan sailkatzeko: aurkariak, lehiakideak, lehiakide indartsuak eta liderrak. Geroz eta eskaintza eta estrategia sendoagoa izan, liderra izatetik geroz eta gertuago egongo da. 

Liderren artean Simpple, Unily eta LumApps daude. Hala ere, nahiz eta lehiakide indartsuen artean sailkatua egon, merkatuko presentzia altuena duen produktua Microsoften Microsoft 365 zerbitzua da. Izan ere, enpresek Microsoft aukeratzen dute \cite{why-companies-microsoft-365} jada ezagutzen dutelako, langileen gailuekin errez integratzen den plataforma osoa delako eta hodeiaren abantailak aprobetxatzen direlako. 

\newacronym{saas}{SaaS}{Software as a Service}

CYC (CYC CONSULTORIA Y COMUNICACIONES SI SL.)\footnote{CYC \url{https://www.cyc.es}} aholkularitza informatikoan aritzen den enpresa da, eraldaketa digitalerako soluzioetan aditua dena. Software aplikazioen eraikuntzan, datuen analisian, teknologiaren aplikazioa industrian eta intranet plataformen sorkuntzan lan egiten da, besteak beste. Modern Workplace eta Modern Workplace Evolution Services departamentuak dira langileentzako plataformen garapena eta jarraipena egiten dutenak eta helburu nagusia enpresetako komunikazioa hobetzea eta lan taldeen produktibitatea areagotzea da. Proiektu guztiek dute oinarri bera: Microsoften SharePoint eta Microsoft 365 \acrshort{saas} zerbitzua.

Microsoft 365 oinarri gisa erabiliz, honek eskaintzen dituen tresnak daude eskura: Office ofimatika \textit{suite} osoa, Outlook posta elektroniko zerbitzua, SharePoint guneak, Teams komunikazio eta kolaboraziorako aplikazioa... eta uneoro berrikuntzak gehitzen dira. Langilearen esperientziak hobetzeko asmoarekin 2021 urtean Viva\footnote{Microsoft Viva \url{https://www.microsoft.com/es-es/microsoft-viva}} tresna sorta aurkeztu zuen Microsoftek erakudeetan detektatu diren arazo ezberdinei erantzuna ematen dietenak: formakuntza, jakintzaren kudeaketa, osasun mentalaren zaintzea... Langileentzako plataformak eraikitzerako orduan, SharePoint guneak dira proiektu guztien oinarria eta guneen gainean konfigurazio eta garapen pertsonalizatuak inplementatzen dira. 

Plataforma hauen arrakasta areagotzeko eta langileei esperientzia intuitibo eta errazak eskaintzeko, gune eta tresnak pertsonalizatu behar dira kasu bakoitzaren kontestu eta egoerara egokituz. Pertsonalizatzeko aukerak anitzak dira: konfigurazio xume baten aldaketa egitetik, plataformaren barnean dagoen aplikazio oso baten eraikuntzara arte. Pertsonalizazio hauen eta eraikitzen diren guneetan dago plataformaren arrakastaren gakoa.  

SharePoint eta Teamsek eskaintzen duen esperientzia natiboa ez da aski izaten enpresen beharrak asetzeko, edo ez gutxienez modu eroso eta erabilgarrian egiteko. Erakundeek hamaika tresna erabiltzen dituzte normalean eta hauen arteko integrazioak ezinbestekoak izaten dira plataforma hauetan. 

%%%%%%%%%%%%%%%%%%%%%%%%%%%%%%%%%%%%%%%%%%%%%%%%%%%%%%%%%%%%%%%%%%%%%%%%
\section{Helburuak}\label{sec:helburuak}
%%%%%%%%%%%%%%%%%%%%%%%%%%%%%%%%%%%%%%%%%%%%%%%%%%%%%%%%%%%%%%%%%%%%%%%%
\newacronym{spo}{SPO}{Sharepoint Online}
Proiektuaren helburu nagusia beharrezko gai teknikoetan ahalduntzea da ondoren bezero errealentzako \acrfull{spo} eta Teamserako utilitateak eraiki ahal izateko. Soluzio hauek bezeroen beharrei erantzun behar diete eta bezeroen egoera zehatzetara egongo dira moldatuak.

Bestetik, departamentuaren parte bilakatzea eta departamentuko lanak bereganatzea da beste helburu nagusia. Helburu honen barne hainbat gaitasun lantzen dira: bezeroen arazoak eta egoera ulertzea eta hauei erantzuna emateko soluzioak sortzea, gerta daitezkeen dudei erantzuna ematea, akatsak araztu eta soluzionatzea, eta ingurune osoaren jarraipena egitea. 

%%%%%%%%%%%%%%%%%%%%%%%%%%%%%%%%%%%%%%%%%%%%%%%%%%%%%%%%%%%%%%%%%%%%%%%%
\section{Plangintza}
%%%%%%%%%%%%%%%%%%%%%%%%%%%%%%%%%%%%%%%%%%%%%%%%%%%%%%%%%%%%%%%%%%%%%%%%
Proiektuaren hasieran ez da emaitza zehatzik definitu, baina behin departamentuan integratzean ezagutuko dira egin beharreko atazak. Ataza hauek \ref{sec:helburuak} ataleko helburuak betetzeko balioko dute. 

Proiektuaren plangintza 2 faseetan banatzen da eta hurrengo lanak burutuko dira:
\begin{itemize}
    \item Microsoft 365ra orientatutako formakuntza
    \begin{itemize}
        \item Web garapena .NET, C\#, jQuery eta Kendu UI erabiliz. 
        \item SharePoint eta Teams pertsonalizatu SPFx, React eta TypeScript teknologiekin.
        \item PowerApps formulario pertsonalizatuak eta Power Automate ekintza automatikoak exekutatzeko fluxuak.
    \end{itemize}
    \item Modern Workplace Evolution Services departamentuan integratzea
    \begin{itemize}
        \item Bezero ezberdinentzako soluzioan diseinatu eta inplementatzea. 
        \item Microsoft 365en inguruko zerbitzua ematea: dudak argitzea eta erroreak konpontzea.
    \end{itemize}
\end{itemize}

Proiektuaren iraupen osoan zehar, honi dagokion txostena idatziko da.

Plangintza osoa eta hilabeteetan zehar egindako lanak \ref{app:gantt} eranskinean aurki daiteken Gantt diagraman irudikatu dira.

%%%%%%%%%%%%%%%%%%%%%%%%%%%%%%%%%%%%%%%%%%%%%%%%%%%%%%%%%%%%%%%%%%%%%%%%
\section{Baldintzen agiria}\label{sec:baldintzak}
%%%%%%%%%%%%%%%%%%%%%%%%%%%%%%%%%%%%%%%%%%%%%%%%%%%%%%%%%%%%%%%%%%%%%%%%
GBL proiektua zuzen garatzeko behar diren baliabide, betekizun eta baldintzak 
deskribatzen dira atal honetan. Garatuko den proiektua web 
teknologietan oinarritutako software tresna baten edo batzuen garapenean datza.

Sortutako soluzioa Microsoft SharePointen, Microsoft 365 ekosistemaren
barruan, oinarritutako aplikazioa, gune sorta edo konponentetaz osatuta egongo 
da. Gerta daiteke soluzioa Microsoft Teams plataformaren barruan egotea edo 
aurten merkaturatutako Viva aukerekin integratzea.

Produktua aurrera eramateko hurrengo puntuak bete behar dira:
\begin{itemize}
    \item Tresnak garatzeko behar diren teknologien formakuntza espezifikoa.
    \item Ordenagailua, pantaila eta bestelako osagarriak.
    \item Garapenerako ingurune osoa: Visual Studio, Visual Studio Code, Postman…
    \item Komunikaziorako plataforma: Microsoft Teams.
    \item Microsoft 365 harpidetza.
    \item Zalantzak eta bestelako ideiak komentatzeko lankidea(k).
    \item Eginbeharren arabera, gai edo eremu zehatzetan aditua den pertsona edo hauek kudeatzen dituena.
\end{itemize}
