%%%%%%%%%%%%%%%%%%%%%%%%%%%%%%%%%%%%%%%%%%%%%%%%%%%%%%%%%%%%%%%%%%%%%%%%
\chapter{Resultados}
%%%%%%%%%%%%%%%%%%%%%%%%%%%%%%%%%%%%%%%%%%%%%%%%%%%%%%%%%%%%%%%%%%%%%%%%

\iffalse
     Kapitulu honetan proiektuan garatutako lanak eta lortutako emaitzak azaltzen dira. Emaitzetara iristeko garatutako lanen definizio funtzionala eta azalpen teknikoak deskribatuz, eta arrazoibidean fokua jarriz.



%%%%%%%%%%%%%%%%%%%%%%%%%%%%%%%%%%%%%%%%%%%%%%%%%%%%%%%%%%%%%%%%%%%%%%%%
\section{Seccion...}
%%%%%%%%%%%%%%%%%%%%%%%%%%%%%%%%%%%%%%%%%%%%%%%%%%%%%%%%%%%%%%%%%%%%%%%%
Praktiken garapenean zehar burututako ekintzak ezagutu eta ulertzea ezinbestekoa da lortutako emaitzak lortzeko prozesua ezagutzeko. 

\subsection{Subseccion}\label{sec:bootcamp}
CYCn erabiltzen den \textit{stack} teknologikoa ezagutzeko, 3 hilabeteko \textit{bootcamp} motako formakuntza proposatu da
garapen inguruneak eta tresnak ezagutu, erabili eta barneratu ahal izateko. Formakuntza 6
kideko taldean burutu egin da, denok enpresan iritsi berriak. 3 zatitan banatu da formakuntza:
lehenik web teknologien oinarriak landu dira, ondoren Microsoft 365 ingurunean murgildu naiz
SharePoint eta Teams tresnetan fokua jarriz eta azkenik Microsoft Power Platformek
eskaintzen dituen Power Automate eta Power Apps \textit{no-code} edo \textit{low-code} tresnak landu dira soluzio
bizkorrak lortzeko. Prozesu honetan departamentu ezberdinetako kideak egon dira tutore
gisa. Atal bakoitza eduki teorikoekin hasi da, ondoren ariketa praktiko batean ikasitakoa
aplikatzeko eta azkenik beste kideekin eta tutorearekin berrikusi dira soluzio ezberdinak
aztertuz, tutoreen \textit{feedback}a lortzeko. 

\subsubsection{SubSubSeccion}
Enpresan lantzen diren proiektu asko \textit{web}an oinarritutako soluzioak izaten dira. Soluzio
hauek eraikitzeko \textit{Frontend}ean HTML, CSS eta JavaScript hirukote ezaguna eta \textit{Backend}ean
Microsoften .NET \textit{framework}a C\# lengoaiarekin batera erabiltzen dira.


\fi